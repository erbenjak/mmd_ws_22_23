\section{Exercise}


\textbf{How can a competitor - in principle - try to steal the valuable data for
recommendation from this website?}

He could try to extract the rating values by liking items and determining the recommended items. With the
main goal of recreating the data from on which those recommendations are calculated in the first place. \\


\textbf{Does this work better when the web shop implemented a content based or a collaborative filtering system?}

This would work better with a collaborative system as the content based one is focused on the preferences of the user, which the competitor would then define himself by the first likes that he chooses. A collaborative system on the other hand draws conclusions from the whole user/item base. \\


\textbf{What data would the competitor be able to infer?}
\begin{itemize}
 \item Clusters of simular items 
 \item List of popular items (recommended quite heavely)
 \item List of unpopular item (rarely recommended)
\end{itemize}


\textbf{Would this technique has an impact on the recommendation system,
i.e., would this attack create a bias on the data?}

Only if it is a small system with a very limited amount of users. Because only than, the impact of single preference is high. This however does not apply when comparing between items. Or when working against a collaborative system. In which case, the attacker would only confuse his own recommendations. \\

\textbf{Why is this attack probably not viable in any case?}

The target system would need to be in a quite extreme dependency for a singular user profile. This would speak against the recommender system heavily, in which case it is probably not even giving good recommendations in the very first place.
